\documentclass{article}

% Language setting
% Replace `english' with e.g. `spanish' to change the document language
\usepackage[english]{babel}

% Set page size and margins
% Replace `letterpaper' with `a4paper' for UK/EU standard size
\usepackage[letterpaper,top=2cm,bottom=2cm,left=3cm,right=3cm,marginparwidth=1.75cm]{geometry}

% Useful packages
\usepackage{amsmath}
\usepackage{amsfonts}
\usepackage{graphicx}
\usepackage[colorlinks=true, allcolors=blue]{hyperref}

\usepackage{xcolor}
\newcommand\SC[1]{{\color{violet}{\it \bf Simon :} #1}}
\newcommand\ET[1]{{\color{red}{\it \bf Emile :} #1}}

\title{Project report : Probabilistic programming}
\author{Emile Trotignon and Simon Coumes}

\begin{document}
\maketitle

\section{Discrete inference by enumeration}

	\subsection{Overview}

	\SC{Draft phase}

	We build our language in Ocaml with a library offering a recursive structure and then allow the use of syntactic sugar to improve use comfort.
	For this first approach, we restrain ourselves to discrete distributions with a finite support. 
	All of this is found in the ``finite'' sublibrary.

	\subsection{A recursive structure}

	Our approach is centered on the introduction of a ``model'' type, which we inductively defined as follows : 


	\begin{equation}
	  Model_1 ::= return \; e \; | \; (assume \; b), \; Model_2 \; | \; a = sample(d) \; in \;Model_2 \; | \;  (factor \; i), \; Model_2
	  \label{eq:1}
	\end{equation}

	With $e$ an expression, $b$ an expression that has type bool, $i$ an expression that has type int, $d$ an expression that has the type of a distribution, and $a$ a variable name.
	We require that every expression be well formed with regard to free variables. ie all of their free variables are bound by the time they are encountered in the tree this forms either by normal Ocaml code or by a sample. \\

	When we move to code, this gives the AST (Abstract Syntax Tree) from figure \ref{fig:AST1}.
	\begin{figure}[h]
	  \centering
	  \includegraphics[scale=0.7]{images/AST1.png}
	  \caption{Type definition for finite distributions}
	  \label{fig:AST1}
	\end{figure}
	
	
	Note that the construction found in our models is existential. \SC{Let Emile word this.}

	Perhaps more interesting, we can see how this can be used to define an example model (see figure \ref{fig:UglyEx}).
	Please note that the bindings happening in samples in equation \ref{eq:1} are replace by functions here, which will be called during evaluation.
	This approach will present multiples advantages and inconvenient in the rest of this project.

	\begin{figure}[h]
	  \centering
	  \includegraphics[scale=0.5]{images/ExampleUgly.png}
	  \caption{A first code example with an unwieldly syntax}
	  \label{fig:UglyEx}
	\end{figure}

	As shown in figure \ref{fig:UglyEx} this code is not very convenient to use. We will see in subsection \ref{subseq:sugar} how we used syntaxic sugar to improve on this.

	\subsection{Recursive evaluation}

	List all cases quickly. Explain main idea of cartesian product. Continuation for sample.	
	\subsection{Syntactic sugar}
	\label{subseq:sugar}

	Yada. Show snippets of example code.

\section{Metropolis Hasting}

	\subsection{Writing continuous distributions}

	A continuous distribution is either a finite distribution or a pair (sample, logpdf).
	We implement only one mode of inference for these distributions, Metropolis Hasting.
	This is found in the continuous sublibrary.

	\subsection{General theory}

	\SC{Re-discuss this first}

	\subsection{We already have continuations}

	Because of the way we wrote the previous logic, it can be followed loosely here while still allowing easy access to continuation.


\section{The tests}

\SC{Not now}

\end{document}
