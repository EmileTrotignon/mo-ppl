\documentclass{article}

% Language setting
% Replace `english' with e.g. `spanish' to change the document language
\usepackage[english]{babel}

% Set page size and margins
% Replace `letterpaper' with `a4paper' for UK/EU standard size
\usepackage[letterpaper,top=2cm,bottom=2cm,left=3cm,right=3cm,marginparwidth=1.75cm]{geometry}

% Useful packages
\usepackage{amsmath}
\usepackage{amsfonts}
\usepackage{graphicx}
\usepackage[colorlinks=true, allcolors=blue]{hyperref}

\usepackage{xcolor}
\newcommand\SC[1]{{\color{violet}{\it \bf Simon :} #1}}
\newcommand\ET[1]{{\color{red}{\it \bf Emile :} #1}}

\title{Project report : Probabilistic programming}
\author{Emile Trotignon and Simon Coumes}

\begin{document}
\maketitle

\section{Discrete inference by enumeration}

	\subsection{Overview}

	\SC{Draft phase}

	We build our language in Ocaml with a library offering a recursive structure and then allow the use of syntactic sugar to improve use comfort.
	For this first approach, we restrain ourselves to discrete distributions with a finite support. 
	All of this is found in the ``finite'' sublibrary.

	\subsection{A recursive structure}

	Our approach is centered on the introduction of a ``model'' type, which we inductively defined as follows : 

	  ** Logic style recursive definition. 
	
	Note that the construction found in our models is existential. \SC{Prettyfy}

	Show snippets of example code. Assure it will be prettier.

	\subsection{Recursive evaluation}

	List all cases quickly. Explain main idea of cartesian product. Continuation for sample.	
	\subsection{Syntactic sugar}

	Yada. Show snippets of example code.

\section{Metropolis Hasting}

	\subsection{Writing continuous distributions}

	A continuous distribution is either a finite distribution or a pair (sample, logpdf).
	We implement only one mode of inference for these distributions, Metropolis Hasting.
	This is found in the continuous sublibrary.

	\subsection{General theory}

	\SC{Re-discuss this first}

	\subsection{We already have continuations}

	Because of the way we wrote the previous logic, it can be followed loosely here while still allowing easy access to continuation.


\section{The tests}

\SC{Not now}

\end{document}
